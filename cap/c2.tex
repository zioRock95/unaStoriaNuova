\chapter{Il regno di Sarcina}
\section{Una strana morte}
\subsection*{21/07/2019}


"Mi spiace tu non sia potuto venire al raduno, è stato divertente!" - "Giusto, il raduno!" fu il primo pensiero del giorno di Benny quando lesse il messaggio di Sarcina sul telefono. Il giovane medico che operava a Roma fu visibilmente stupito dall'assenza di messaggi molesti sul gruppo. Cercò tra i membri: nessun bannato. Non fece in tempo a scendere di casa che il suo telefono squillò: era Gianluha.

"Mi serve una mano, Benny." - disse - "E' successo qualcosa di molto strano questa notte, mi serve il tuo parere da medico. E' una cosa seria. Puoi venire? Ti pago.", Benny sentì un brividio percorrergli tutta la schiena e la gola annodarsi, sospirò un secondo - "Ci sei?" - tentennò ancora un attimo e poi rispose: "Devo prendere un treno, davvero? E' sabato, non lavoro, ma avevo impegni personali. Quanto è seria la questione?". 
Si percepì il nervosismo di Sarcina che dopo un stante tuonò: "Te lo dirò seriamente, prendi fiato, non svenire. Te lo dico è una cosa brutta, preparati: è morto un ospite." - Benny trattenne le emozioni il tempo esatto per dire che sarebbe corso a Firenze, poi attaccò il telefono e si sfogò un attimo.

Il medico fece subito i biglietti in stazione e si mise a leggere un libro aspettando che il suo treno fosse pronto sul binario. Circa un'ora dopo si accomodò sul suo posto e scrisse i dettagli del viaggio a Sarcina via messaggio, chiedendo anche se sapesse qualcosa sul silenzio che regnava in AJG.
La risposta fu pressoché immediata: "Troverai un auto pronta ad aspettarti in Santa Maria Novella. Per quanto riguarda AJG non so molto, ieri li ho lasciati tutti ubriachi, staranno ancora dormendo".
Pensò che effettivamente erano solamente le 10 di mattina in un sabato qualsiasi, si mise l'anima in pace e tornò a leggere "*Titolo che potrebbe tornare utile come elemento*".

Un paio d'ore dopo il treno annunciò il suo arrivo alla stazione fiorentina, prese la sua piccola valigia e aspettò lo stop del treno sulla porta di uscita. Appena scese dal mezzo notò un bagliore insolito in cielo, ma non ci diede molte attenzioni. Cercò subito di rintracciare l'auto che lo stava aspettando, fu abbastanza immediato: c'era solo un veicolo fermo sulla banchina, tutti gli altri sembravano disperatamente cercare parcheggio.
Il viaggio fu privo di parole, Benny ne approfittò per continuare a leggere, passarano una decina di minuti quindi l'auto si fermò in doppia fila: non c'era parcheggio. 

"Benny, caro, sono Gianluha!" - Sarcina lo aspettava sull'uscio della porta, invitandolo con caldi gesti ad avvicinarsi - "Vieni, ho preparato due spaghi per pranzo".
Il clima sembrava totalmente diverso, il medico tentennò un po', poi si fece coraggio e si accomodò nella casa di Sarcina.

\section{La sindrome del parcheggio}

Il pranzo passò felicemente discutendo del più e del meno, Benny trovava il pasto ottimo e Sarcina intratteneva il suo ospite con un fare molto professionale. [Ampliare cosa del pranzo per definire le psicologie]
Il pasto si concluse e con aria tetra Benny cercò di far capire a Gianluha le sue intenzioni: la curiosità iniziava ad aleggiare forte in lui.

Sarcina portò il medico nella stanza [descrizione stanza e accenno al ciccione addormentato] e chiuse la porta dietro di lui, quindi con voce sicura gli puntò un coltello nella schiena e disse soavemente a benny: "Ho ucciso Mancini con questo preciso coltello e Federica è nelle mie segrete. Salvo le fa compagnia nella cella di fianco, devi decidere se stare con me o fare coppia con Mancini." - Benny cauto disse: "Non ci vedo molta possibilità di scelta, cosa vuoi da me?"
Gli sguardi si incrociarono e il medico capì di essere ormai destinato ad assecondare Gianluha. "Starà mentendo su Mancini? Che cosa è successo ad AJG?", non fece neanche in tempo a porsi mentalmente queste domande che istericamente Sarcina esclamò: "Aiutami a portare questo tossico ciccione al piano di sotto, c'è una sorpresa che ti aspetta."

La fantasia di Benny iniziò a vagare in ogni direzione, non riusciva a tenere un pensiero per più di qualche secondo, dei brividi gli percorsero tutta la schinea, si sentiva estremamente euforico ed era una sensazione che non provava da tanto. I due presero il malcapitato ospite che non accennò mezzo movimento, "Ma è vivo si?!" - chiese con impeto il medico, "Si, è solo anestetizzato, ti sento determinato e ancora devi vedere la parte più bella, è un'ottima cosa, sarebbe veramente un peccato doverti fare la loro stessa fine."

Scese tre rampe di scale l'ambientazione era totalmente diversa [descrizione ambientazione], Sarcina si sentiva nel suo ambiente naturale e quasi danzando portò alla luce fatti orribili: da solo aveva praticamente finito un macchinario per il controllo del cervello umano, la cantina di quel palazzo era in realtà un laboratorio con tecnologia mai vista prima. 
Il medico pensò seriamente di trovarsi davanti ad un genio del male, non riusciva però a chiedersi cosa volesse ottenere davvero Gianluha dal mondo, fece per chiederglielo, ma appena alzò lo sguardo tutto gli venne più chiaro: un cartellino esposto sotto un cadavere diceva "Mancini ha trovato parcheggio al Franchi, inammissibile: 20/07/2019" - c'erano molti altri ganci simili a quelli che tenevano sospeso mancini, tuttavia vuoti. 

"Ci appenderò presto Sirio e Rora" - Benny sentì Gianluha rimuginare tra sè e sè - "Cosa?!" chiese di impatto, gli si gelò il sangue: doveva essere seriamente successo qualcosa di importante al raduno. "Niente, niente, scherzo, chi li ha mai visti quei due. Piuttosto, dovresti iniettare questa soluzione nel tossico, se dovesse morire ti chiederei di portarlo in ospedale, eseguendone l'autopsia."
Il medico perplesso: "Non penso di avere tutto qu", fu interrotto subito da Sarcina che un ghigno disse: "Non devi pensare a niente, fidati di me."

Gianluha passò una siringa piena di una strana soluzione arancione a Benny e con un cenno lasciò intendere che doveva procedre. La paura nel medico sparì non appena puntò l'ago sulla schiena della povera cavia, la professionalità non gli fece sbagliare nulla, svuotò la siringa dentro il tossico, dopodiché sospirò notando che non stava succedendo nulla. "Posso fumare una sigaretta?" domandò il medico guardando da uno schermo l'entrata della casa - "No, no, qua non puoi. Dopo." disse Sarcina mentre cercava qualcosa tra mille fogli. 
Il medico notò un cambio di luce abbastanza forte nel video, che fosse sparito quello strano bagliore d'orato?
Non riuscì a pensare ad altro perché udì un rumore provenire da dietro di lui: si era svegliato il tossico. Non spiccicava parola, non era perplesso, curioso, non era niente. Occhi vitrei, pieni di emozioni, sembrava fosse mutato. "Ah, ora se lo liberiamo cercherà a tutti costi un auto e andrà a cercare parcheggio intorno al Franchi!" esclamò Gianluha con un tono malefico.
"Tuttavia questi occh... che sia andato sbagliato qualcosa?" - si domandò il pazzo tra sè e sè - "Benny!" - con voce più alta - "Tieni. Queste sono le formule per il siero, dimmi cosa c'è di sbagliato, perché non ha gli occhi assenti? Secondo te è sotto shock o è andato tutto per il verso giusto?"

Il medico fu incaricato di controllare il tossico nel Mondo reale, quindi lo liberarono dalle catene e lo portarono alla macchina di Gianluha. "Dimmi se noti qualcosa di strano?" disse Gianluha fiducioso "e prova a sparire e verrò personalmente ad ucciderti. So dove ti trovi in qualsiasi momento."
Benny pensò fosse una mera intimidazione, ma dopo qualche minuto gli arrivò un messaggio di posizione cocn scritto: "La tua posizione è questa." ed era incredibilmente esatta. Pensò fosse una localizzazione dell'automobile, ma non si fidò a provarlo in quel momento.

Dopo ore di giri attorno allo stadio il tossico sembrava assolutamente non affranto dalla cosa, non frustrato, niente. "Ordina al tossico di levarsi, e guida tu verso casa." diceva un messaggio da Gian, lui lo fece, in pochi minuti tornarono a quello che da quel giorno in poi sarebbe divenuto il quartier generale di un impero nascituro.

\section{Il risveglio}
\subsection*{22/07/2019}
[Benny si accorge che manca una componente fondamentale nel siero: la pazzia di Sarcina, estrae quella parte di DNA e rifà la pozione asserendo che non avrebbe funzionato su chi già infetto, riesce quindi a preservare il tossico (pola) in quello stato] 
[Sarcina va a provare l'effetto del nuovo Siero mentre Benny chiede di riposarsi per andare in chiesa all'alba del giorno dopo per confessarsi al signore]


\subsection*{23-26/07/2019}
[Qua si parla dei test che conduce Benny]
Gli sguardi incrociati (elemento 2) con gli altri assenti [non troppo dettaglio]


\subsection*{Nelle settimane a seguire}
[Il siero viene provato su nuovi ospiti e funziona a meraviglia, Sarcina inizia quindi a cercare di arrivare a persone importanti per avere un esercito di zombie al comando, vuole conquistare Firenze]

[In poche mosse i due riescono ad avere un boost di recensioni sull'hotel e arrivano ad avere il monopolio sulla città]


\section{Un messaggio da Firenze}

\subsection*{23/07/2019}

incontro con la sacerdotessa nella chiesa di santa maria diffiori

\subsection*{28/07/2019}

Rora scappa con Pola verso luoghi più sicuri, in Taxi (luca)
