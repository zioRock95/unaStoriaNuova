\chapter{Il regno di Sarcina}
\section{Una strana morte}
\subsection*{21/07/2019}


"Mi spiace tu non sia potuto venire al raduno, è stato divertente!" - \textit{Giusto, il raduno!} fu il primo pensiero del giorno di Benny quando lesse il messaggio di Sarcina sul telefono. Il giovane medico, a quei tempi ancora impiegato a Roma, rimase sinceramente stupito dall'assenza di messaggi molesti sul gruppo. Controllò i membri: nessun bannato. Non fece in tempo ad uscire di casa che il suo telefono squillò: era Gianluha.

"Mi serve una mano, Benny." - disse - "E' successo qualcosa di strano stanotte, mi serve il tuo parere di medico. E' una cosa seria. Puoi venire? Ti pago."

Benny sentì un brividio percorrergli tutta la schiena e la gola annodarsi, sospirò un secondo.
"Ci sei?"

Tentennò ancora un attimo, poi rispose: "Devo prendere un treno, davvero? E' sabato, non lavoro, ma avrei impegni personali. Quanto è seria la questione?"

Il tono di Sarcina mutò improvvisamente, la sua voce si era fatta stranamente nervosa: "Te lo dirò onestamente, prendi fiato, non svenire: è morto un ospite."

Benny trattenne a stento le emozioni: "Arrivo", rispose, poi attaccò e si sfogò imprecando.

Il medico fece subito i biglietti in stazione e si mise a leggere un libro su una panchina del binario aspettando l'arrivo del suo treno.

Circa un'ora dopo si accomodò sul suo posto ed inoltrò i dettagli del viaggio a Sarcina, chiedendo spiegazioni sull'insolito silenzio che regnava in AJG.

"Troverai un auto pronta ad aspettarti in Santa Maria Novella", rispose quasi immediatamente Sarcina. "Per quanto riguarda il gruppo cosa vuoi che ne sappia, ieri li ho lasciati tutti ubriachi, staranno ancora dormendo".

\textit{Massì, sono le 10 di un sabato mattina qualsiasi, staranno dormendo.} Benny si mise l'anima in pace e tornò a leggere "*Titolo che potrebbe tornare utile come elemento*".

Un paio d'ore dopo il treno annunciò il suo arrivo alla stazione fiorentina; il medico afferrò la sua ventiquattrore e attese sulla porta d'uscita lo stop del treno. Appena sceso non potè fare a meno di notare un insolito bagliore dorato nel cielo: \textit{Me la ricordavo diversa la nebbia} commentò ironicamente.

Rintracciare l'auto che lo stava aspettando fu abbastanza immediato: era l'unico veicolo fermo sulla banchina, attorno era tutto un cercare disperatamente un parcheggio.

Durante il viaggio non volò una mosca, e Benny ne approfittò per continuare a leggere; passarono una decina di minuti quando infine l'auto si fermò, ma in doppia fila: non c'era parcheggio.

"Benny, caro, finalmente ci conosciamo!" - Sarcina lo aspettava sull'uscio della porta, invitandolo con caldi gesti ad avvicinarsi.

"Vieni, ho preparato due spaghi per pranzo".

Il clima sembrava totalmente diverso, il medico tentennò un po', poi si fece coraggio ed entrò.

\section{La sindrome del parcheggio}

Il pranzo era già pronto, e il padrone di casa lo invitò ad accomodarsi.

\textit{Strano}, pensò Benny, \textit{mi ha portato qui perché gli è morto un ospite ma ora pensa a mangiare, ma cos'ha in quella testa.}

Sta di fatto che il pranzo passò tranquillamente discutendo del più e del meno: Benny trovò il pasto ottimo e Sarcina, rotto il ghiaccio, iniziò ad intrattenere il suo ospite col tipico fare per cui era famoso sul gruppo.

"Sempre peggio sta città maledetta, non si trova un parcheggio manco a pagarlo. Ogni pomeriggio mi tocca fare i chilometri per mettere la macchina da qualche parte e tutto per poter andare casa mia, mica in vacanza, a casa mia; quando gioca quella squadraccia poi ancora peggio, prima o poi lo distruggo sto posto."

Benny rimase interdetto, si sarebbe certo aspettato che Gianluha potesse essere il solito personaggio anche nella vita reale, ma c'era qualcosa nella sua voce che non quadrava.

Il pasto si concluse e con aria inquieta Benny cercò di capire le intenzioni dell'amico: una strana curiosità stava iniziando a farsi strada nella sua testa.

"Va bene dai, seguimi, ti faccio vedere come mai siamo qui."

Sarcina attraversò il corridoio, fermandosi in corrispondenza della seconda porta a sinistra.

La stanza in cui entrarono era riscaldata ma molto disordinata, con mucchi di scartoffie sulla scrivania; in un angolo giaceva un corpo su una branda, evidentemente il morto per cui era lì.

Sarcina chiuse la porta, poi si avvicinò all'amico e gli mise una mano sulla spalla; Benny stava per chiedergli il da farsi, quando l'altro estrasse di colpo un coltello puntandoglielo alla gola.

"Ho ucciso Mancini con questo preciso coltello, Benny", gli disse con voce paurosamente calma, "e Federica è nelle mie segrete. Salvo le fa compagnia nella cella di fianco, decidi tu se stare con me o fare coppia col romano."

\textit{Non ci voglio credere, non può essere vero. Che cazzo è successo ad AJG?}

"Mi pare di capire che non ho molte alternative, cosa vuoi da me?", rispose cauto il medico.

"Aiutami a portare questo tossico ciccione al piano di sotto, c'è una sorpresa che ti aspetta."

La mente Benny iniziò a vagare in ogni direzione, non riusciva a tenere un pensiero per più di qualche secondo; dei brividi iniziarono a percorrergli la spina dorsale ma si sentiva estremamente euforico, una strana sensazione che non provava da tanto.

I due sollevarono il malcapitato, che non accennò mezzo movimento.

"Ma è vivo?!" chiese con impeto il medico.

"Si, è solo anestetizzato, ti sento determinato e non hai ancora visto la parte più bella, è un'ottima cosa, sarebbe veramente un peccato doverti far fare a meno di te."

Scese tre rampe di scale l'ambiente era totalmente diverso.

Sarcina lo portò in una stanza fredda, umida, per certi versi quasi trascurata: Benny, entrando, scansò inorridito persino una scolopendra morta sul pavimento, probabilmente uccisa proprio in quel punto e neanche tolta di mezzo.

Sarcina invece dava l'idea di sentirsi nel suo ambiente naturale e quasi danzando portò alla luce ciò a cui in realtà stava lavorando: "Caro amico, ti presento il mio gioiello!"

La parte destra della stanza sembrava il set di un film fantascientifico: un lettino da ospedale era installato su una pedana rialzata di un buon mezzo metro, con due scalini che permettevano di raggiungerlo; tutto attorno era un groviglio di cavi, tubi, schermi e tastiere.

\textit{Pare un film tipo Divergent, manca solo la cavia}, pensò Benny tra sé.

In tutto questo, però, il medico non riusciva a capire quale fosse l'obiettivo di Gianluha.

Fece per chiederglielo, ma appena rivolse lo sguardo dell'altra parte della stanza tutto gli sembrò più chiaro: un cartellino appiccicato al muro, sopra a quello che sembrava a tutti gli effetti il cadavere penzolante di Mancini, recitava:

\begin{lstlisting}
20/07/2019
Ha trovato parcheggio al Franchi
Inammissibile
\end{lstlisting}

Benny cadde nello sconforto: Sarcina non aveva mentito, e una delle colonne portanti di quel gruppo era davvero finita appesa ad un muro.

Molti altri fogli vuoti erano attaccati alle pareti, e Benny sperò con tutto il cuore che non dovessero mai ospitare le lettere del suo nome.

"Ci appenderò presto Sirio e Rora", sentì borbottare l'altro tra sé e sé.

"Cosa?!" chiese Benny d'impatto, il sangue che gli si era gelato nelle vene: doveva essere seriamente successo qualcosa di importante al raduno.

"Niente, niente, scherzo, neanche si sono presentati quei due, chissà che fine avranno fatto. Piuttosto, devi iniettare questa soluzione nel tossico; dovesse morire portalo in ospedale e fagli un'autopsia"

Il medico era perplesso: "Non penso di avere tutt-"

"Non pensare a niente" lo interruppe Sarcina con un ghigno, "fidati di me."

Gianluha passò una siringa piena di uno strano liquido arancione a Benny e gli fece cenno di procedere.

La paura nel medico sparì non appena puntò l'ago sulla schiena della povera cavia, da lì lo guidò la professionalità: svuotò la siringa nel corpo del tossico, dopodiché sospirò notando che non stava accadendo nulla.

"Posso fumare una sigaretta?" domandò guardando da uno schermo l'esterno della casa.

"No, no, qua non puoi. Dopo." rispose Sarcina mentre cercava qualcosa tra i suoi fogli.

Benny notò un cambio di luce decisamente forte nel video: \textit{Che sia sparito quello strano bagliore dorato?}, si domandò dubbioso.

I suoi pensieri si arrestarono quando udì un rumore dietro di lui: il tossico si era svegliato.

\section{Il risveglio}

Non spiccicava una parola, non era perplesso, curioso, spaventato, non era niente.

"Ah, ora se lo liberiamo cercherà a tutti costi un'auto e andrà a cercare parcheggio intorno al Franchi! Se tutti hanno questa smania di smentirmi, allora che ci provino pure in eterno!" esclamò Gianluha con un tono malefico.

\textit{Tuttavia questi occh... che sia andato sbagliato qualcosa?} si domandò il pazzo tra sé e sé.

"Benny! Tieni, queste sono le formule del siero, dimmi cosa c'è di sbagliato.

Perché non ha gli occhi assenti? Secondo te è sotto shock o è andato tutto per il verso giusto?"

Il medico fu incaricato di controllare il tossico nel mondo reale: lo liberarono dalle catene e lo portarono alla macchina di Gianluha.

"Dimmi se noti qualcosa di strano" si raccomandò, "e ricorda: prova a sparire e finirai a fare compagnia a Mancini. So dove trovarti."

Benny pensò fosse un'intimidazione, ma dopo qualche minuto Sarcina gli inoltrò la sua esatta posizione: "Te l'ho detto. So dove sei."

Pensò stesse usando il gps dell'auto, ma al momento aveva ben altro a cui pensare.

Passò ben tre ore col tossico, osservandolo cercare invano un parcheggio attorno allo stadio e non demordere, non affliggersi, non tradire la minima frustrazione davanti all'impossibilità della cosa.

"Ordina al tossico di levarsi, e guida tu verso casa." diceva un messaggio di Gian.

Benny eseguì, ed in pochi minuti tornarono a quello che da quel giorno sarebbe divenuto il quartier generale di un impero.

La serata praticamente volò. Benny non ricordò altro oltre alla cena: una pasta all'amatriciana squisita.

\subsection*{22/07/2019}

\textit{Buongiorno un cazzo, però cucina bene Sarcina, chissà che ci mette nei piatti}, Benny si alzò dal letto stranamente riposato, porse lo sguardo sul soffitto, un raggio di sole tagliava completamente in due la stanza precisamente sopra di lui. Si tirò su, a casa sua era solito farsi un caffè appena svegliato e poi uscire per fare colazione al bar dell'ospedale. Passarono pochi istanti e si sentì Gianluca urlare dal piano di sotto \textsc{Benny muoviti! Nullafacente del cazzo, che c'è da lavorare.}

\textit{Oddio ma è già sveglio}  - \textsc{Inoltre è pronta la colazione!}

Il medico si alzò dal letto e notò il soffitto rosso a pois bianchi \textit{Boh, pare una amanita muscaria} - uscì dalla sua camera e si ritrovò in un'anti sala circolare, con 5 porte equidistanti tra loro, al centro una scala chiocciola. \textit{Spero di non morire mentre scendo per il gambo di sto fungo}, scese le scale si ritrovò direttamente in una sala da pranzo deserta.
[aggiungere descrizione ambienti]

"Dai! Ovviamente la colazione non è pronta proprio per niente, godo. Io c'ho da lavorare dai, arrangiati!", Gianluca uscì quindi da una porta dall'altra parte della sala e la chiuse a chiave.

Un po' spaesato cercò le cose per farsi un caffè, trovò una moka nello scolapiatti e, con non poca fatica, trovò anche il caffè.

\textit{Ma un po' di musica almeno, qualcosa, ma poi che cazzo devo fare boh} - Mentre il caffè veniva su andò a cambiarsi in stanza. Alzò la tapparella e notò un foglio lasciato sulla scrivania:
\begin{lstlisting}
Perfeziona il siero
Nella stanza numero 2 trovi i fogli che ti ho fatto vedere ieri
Nella cena di ieri c'era del sonnifero, dovresti essere riposato
\end{lstlisting}

\textit{Lurido figlio di puttana, del sonnifero. Beh , grazie, effettivamente sono riposato.} - Prima di scendere in cucina passò per la camera 2 e diede un rapido sguardo ai fogli. \textit{Cazzo, il caffè!}

Benny volò in cucina, quasi ammazzandosi per le scale, ma si fermò appena sentì un forte odore di gas. Guardò tutta casa con un ghigno malefico \textit{Oh si, potrei farlo!} - balzò quindi sul piano cottura e notò il caffè riversato ovunque e la fiamma totalmente assente.

\textit{Devo rifarlo. Non mi va, ma chi cazzo usa ancora la moka nel 2019, vabbè} - prima di tornare al piano di sopra aprì una finestra.

\textit{Vediamo cosa c'è da fare qua} - Il medico indossò le cuffie ascoltando "Burning Pipers Hut" di Beltaine - \textit{*considerazioni sulle cose*, che razza di imbecille, vuole rendere le persone pazze quanto lui senza metterci direttamente la sua pazzia. Sembra tutto ok altrimenti, *qualcosa* c'è, *again* anche. Forse si potrebbe cambiare il sangue con una soluzione diverso, potrebbe diventare un problema con i gruppi sanguigni. Fatto sta che devo aspettarlo, fumerò una sigaretta.}

Benny si fumò tranquillamente la sua sigaretta, mentre ascoltava del buon Jazz, e decise quindi di andare a curiosare nell'altra parte di casa, lì dove Sarcina era sparito circa un'ora prima. \textit{Ma vuoi vedere che non lavora un cazzo, quello si ammazza di seghe in streaming.} nel mentre Benny scrutava la casa dall'esterno \textit{ma come cazzo arrivo io di là? Che poi è tutto spento} non fece in tempo a finire il pensiero che, provando ad aprire la porta d'ingresso, entrò in casa di Gianluha. \textit{fior e fior di inferiate, e poi lascia la porta aperta. Che problemi ha?!} - \textsc{"Chi cazzo sei tia mazo!"} - "Ma ti ripigli coglione, stavi anche dormendo. " la rabbia montò nel medico "Cioè io devo svegliarmi presto per lavorare e te dormi?! Mi serve il tuo fottuto DNA per finire il tuo lavoro."

"Senti, intanto ti calmi." - rispose scazzato Sarcina - "Io sono rimasto sveglio tutta la notte, a differenza tua. Lavoravo."

- "Ma brutta faccia di merda" - lo fulminò Benny - "grazie al cazzo che dormo se mi addormenti. Che sonnifero hai usato?"

"Cosa vuoi che cambi, piuttosto, a cosa ti serve il DNA, perché?"

- "Ma... " - \textit{se non lo ammazzo guarda!} - "Mi hai chiesto di ultimare il tuo siero. Per farlo mi serve il tuo DNA. Per cosa pensi che mi serva?"

"Ma io che cazzo ne so scusa (messaggio inoltrato)" - \textit{ma questo è veramente un meme oddio non lo fa apposta!} pensò il medico.


[Benny si accorge che manca una componente fondamentale nel siero: la pazzia di Sarcina, estrae quella parte di DNA e rifà la pozione asserendo che non avrebbe funzionato su chi già infetto, riesce quindi a preservare il tossico (pola) in quello stato]


\subsection*{23-26/07/2019}
[Qua si parla dei test che conduce Benny]
Gli sguardi incrociati (elemento 2) con gli altri assenti [non troppo dettaglio]


\subsection*{Nelle settimane a seguire}
[Il siero viene provato su nuovi ospiti e funziona a meraviglia, Sarcina inizia quindi a cercare di arrivare a persone importanti per avere un esercito di zombie al comando, vuole conquistare Firenze]

[In poche mosse i due riescono ad avere un boost di recensioni sull'hotel e arrivano ad avere il monopolio sulla città]


\section{Un messaggio da Firenze}

\subsection*{23-26/07/2019}

[Benny nota lo stato dei membri di AJG su telegram e vede che Rora entra sempre alla stessa ora, le scrive e lei dice di andare a cercare la verità confessandosi col signore a Santa Maria dei Fiori]
\subsection*{06/08/2019}

Benny si svegliò e si diresse verso Santa Maria dei Fiori, non era solito per il medico andare in chiesa, anzi a dirla onestamente la trovava una cosa piuttosto insensata, ma pur di liberarsi da tutto ciò che stava succedendo avrebbe fatto di tutto. Il centro era più vuoto del solito, le conseguenze della macchina del pensiero erano palesi in città. Non ci mise tanto tempo Benny a raggiungere Santa Maria dei Fiori, tentennò un po', pensò tra sè e sè se fosse poi veramente intenzionato a tradire Sarcina, o se potesse in qualche modo riuscire a liberare la città da questa piaga. Si fece coraggio ed entrò, la chiesa era vuota, si udivano rumori provenire dalla sagrestia, ma Benny lasciò perdere e andò al battistero a parlare a tu per tu con il crocifisso, non sapeva se si facesse così, ma l'assenza di persone lo portò a non curarsi troppo di ciò.**con che italiano figa vabbe**

"Rora dimmi che ci sei, dammi un solo segnale, sono io, sono Benny. Sarcina dorme, salvami." - dalla sagrestia si udirono dei passi venire verso l'aracata principale, poi il silenzio, "Non so di chi tu stia parlando, ma se ti è stato detto di cercare la verità in questo posto, così deve essere." - da dietro l'organo compare una donna discretamente alta, maschera sul volto, capelli e corpo nascosto da una veste elegante di almeno 200 anni fa - "Ti è stato detto di venire in questo luogo?"
Benny sospirò, pensò che doveva essere allucinato e che Rora non si fidasse di lui pertanto non si fosse presentata, le sue speranze crollarono e in un pianto interminabile mentre spiegò alla figura cosa stava succedendo nella città.
"Ciò che dici non mi è nuovo, il bagliore l'ho notato anche io, le scritture parlano chiaro, solamente chi ha visto in faccia la pazzia potrà liberare l'umanità da questa piaga."

Benny si sentì svenire, aveva bisogno di più informazioni per capire quelle parole, "Com'è possibile che qualcuno abbia visto in faccia la pazzia? Gianluha è impazzito da tempo, non è una cosa che si nota più nei suoi occhi, fin dal primo giorno ha quelle espressioni."
La figura fece per andare via in silenzio, ma udendo il singhiozzare di Benny si fermò, si girò e disse: "Non è chiaro che posizione tu abbia in questa guerra, ma devi essere deciso e schierarti dal lato del bene o dal lato di Gianluha. Tutto ciò che devi fare è portare in questo posto sacro una persona che può essere diversa da tutte le altre, tutto ciò che devi fare per me, per il bene, per l'umanità, se ti è veramente a cuore aiutare, è portare a me il prescelto, digli di chiedere della sacerdotessa. Del resto me ne occuperò io."

"E' questo quello che vuoi?" - urlò disperatamente Benny - "Mi state facendo impazzire, mi avete stufato tutti. Cosa succede in questo Mondo?" - le forze iniziavano a venirgli meno, così come la voce - "Basta, vi supplico, basta. Smettetela di trattarmi come un burattino. Io ora porto qua l'unica persona diversa che ho visto in quest'ultimo mese e te lo fai andare bene, uccidimi se sbaglierò, esponimi come trofeo così come fa quell'altro pazzo." - su queste parole Benny si voltò e con un passo stanco, ma deciso, uscì dalla chiesa e tornò verso il quartier generale, intercettò il primo tossico che aveva trattato e gli ordinò di andare in quella chiesa.

Quando il tossico aprì le porte e con una voce spenta disse: "Cerco la sacerdotessa." - sguardo perso nei confronti del crocifisso che non accennava a proferir parola - "Cerc", le sue parole furono interrotte dalla medesima figura di prima che, con passo frettoloso, si avvicinò al tossico ed esclamò decisa: "Via da questa sala, seguimi, rapido."
La sacerdotessa si incamminò verso la sagrestia con fare sbrigativo ed il tossico ordinatamente la seguì, passarono diverse sale, scesero un paio di rampe di scale, si ritrovarono quindi in una sala che avrà avuto circa 500 anni. "Qui è dove io continuo a capire cos'è successo al mio amato." - il tossico la guardava con occhi lucidi, ma senza alcun movimento, nè suono - "Qui è dove io cerco di capire se mai avrò la possibilità di parlare con Mancini."

Al tossico si mosse leggermente una pupilla, non sfuggì alla sacerdotessa, che, tradita dall'emozione, si levò la maschera. Capelli scuri a caschetto fino alle spalle, occhi marroni, fisico minuto.
"Puoi riconoscermi?" - disse guardando negli occhi il tossico: non accadde nulla - "Mancini?" - la pupilla ebbe un leggero movimento di nuovo - "Sei tu. Sei veramente tu." - la sacerdotessa scoppiò in lacrime abbracciando il povero tossico - "Cosa ti hanno fatto? Cos'è rimasto del tuo corpo? Mi vendicherò, ti avrò di nuovo." - non successe nulla.

"Mettiti quei vestiti." ordinò la sacerdotessa al tossico. Mentre lui si cambiava lei si mise nuovamente la maschera e poi disse con un tono riflessivo: "Devo trovare il saggio, ti prometto che lui potrà aiutarci." - cambiò quindi il tono di voce e con fare autoritario continuò - "Seguimi, per sempre, seguimi da ora, replica ogni mio gesto.".
I due uscirono dalla chiesa e, con non poca fatica, riuscirono a salire su un taxi. L'odore era putrido, il tassametro segnava più di 3000euro, il tassita non aveva gli occhi spenti, ma non proferiva parola, la sacerdotessa parlò quindi al tossico: "So che non puoi parlare, ma ti prometto che cambieremo tutto. Cambierà tutto insieme. Devo solamente trovare lui." - il tassista accelerò improvvisamente e cambiò totalmente rotta - "Cosa fai?!" disse la sacerdotessa seccata, ma non ottenne risposta.

Dopo pochi minuti di corsa rallentò nuovamente, si girò e con un tono flebile si spiegò: "Sei la seconda ed unica persona che parla che sento da un mese. Io so chi stai cercando, o almeno credo, l'ho portato in aeroporto qualche giorno fa. Ha lasciato un bigliettino, spero potrà aiutarti. Il tassista indicò la tasca dietro al suo sedile, poi disse "Non puoi andare in aeroporto con quel soggetto. E' riconoscibile, tu non sarai sicuramente niente, ma lui è Pola, del gruppo AJG. Era una persona pressoché uguale ad ora, non capiva niente e parlava poco anche prima. Se mai tornerà in se digli che Luca ha provato a salvargli la vita."

La sacerdotessa ebbe la fortuna di indossare la maschera che coprì una lacrima che scese rigando tutto il viso - "Grazie, ma come lo raggiungerò?" - lui fiero riprese la guida e rispose: "Ti porterò fuori da questa città, dopodiché ti incamminerai ad Est, a piedi. Sono successe cose strane dopo un giorno di metà Luglio, ho perso tutto, ma non la convinzione che questa piaga finirà, prima o poi. Vai ad Est, cammina con Pola al tuo fianco, muori per lui se necessario, incontrerai chi di dovere, sono sicuro che risolleverete di nuovo la città da questa piaga."
Un'ora dopo i due si salutarono, e la sacerdotessa ringraziò Luca - "Ti ricorderò per sempre, te cosa farai?" - lui guardò Pola negli occhi e disse - "Io resto, per cercare di capire di più. Se avrai modo di usare il telefono in futuro, mi chiamo Luca\_Po. Se mai ti servirà aiuto, chiamami, non scrivere." - alzò il finestrino e andò via, senza dire nulla.

I due si incamminarono tra le colline.
