\chapter{La rovina del Mondo}
\section{Il Club di AJG}
	\subsection*{Italia, 2019}

I primi segni lasciati sul mondo da Sarcina risalgono a questo periodo, agli albori del terzo millennio, e ci giungono da fonti vicine al "Romano", Gianmarco Mancini.

Era un'età fiorente quella, un'età fatta di goliardia e meme brutti tanto da guadagnarsi l'appellativo di Era dello Zitto Coglione, ma nessuno avrebbe mai pensato che l’umanità stesse invece avviandosi verso il millennio più oscuro mai documentato.

Si afferma che agli inizi di quell'anno iniziò a prendere piede, tra le altre, l'apparentemente innocua usanza di prendere continuamente in giro Gianluha Sarcina, fiorentino, per la pressoché totale assenza di parcheggi vicino casa sua: abitava infatti vicino allo stadio Artemio Franchi, più o meno un chilometro, a esagerare sarà un chilometro e mezzo, e specialmente nei giorni in cui era in programma una partita casalinga della Fiorentina trovare un posto in cui posteggiare diventava praticamente impossibile.

Ai tempi Sarcina era una persona simpatica, a capo di un gruppo virtuale di 1000 povere anime, il Club di AJG, di cui facevano parte anche Mancini e un suo vecchio amicod'infanzia, Sirio, che però aveva sempre preferito non mostrare il proprio volto e la propria faccia ai membri.

Nel gruppo capitava di assistere sia a scherzi che a momenti incresciosi: spesso si vedeva gente perdere il senno dietro ad individui femminili dalla dubbia moralità, discriminazioni territoriali ai danni di persone provenienti dal sud del Paese ma anche dell'emisfero, ed un continuo invio di organi genitali chiaramente bisognosi di interventi chirurgici, ma alla fine tutti si trovavano d'accordo non appena si inviassero dei meme che accostavano Gianluha a Toad, il famoso fungo antropomorfo di Super Mario, a causa della sua oggettivamente buffa capigliatura.

Sarcina, comunque, non se la prendeva: era un amministratore degno del nome, permissivo, scherzoso e per questo ben voluto.

Nessuno si accorse di nulla ai tempi, ma c’erano tutti i presupposti per il peggio già allora.

E' in questo periodo che si hanno le ultime certezze sulla vita degli umani prima del grande buio.

Il club di AJG rappresentava l'ultima alternativa sociale al degrado della musica Trap che ormai aveva preso il sopravvento: La Proposta della Notte, una rubrica serale dal dubbio gusto, rappresentava uno dei momenti in cui si palesava la forte gelosia di Sarcina nei confronti di Federica, una ragazza simpatica e già fidanzata, che non si era mai fatta intimorire dalle molestie dei vari membri del club.

Questa, infatti, era solita ridere agli scherzi di Mancini, scatenando la rabbia dell'admin che si vedeva scavalcato nelle preferenze dell'amica.

Molte discussioni, poi, erano animate da Salvo: un ragazzo compreso da tutti, ma non troppo ben voluto a causa della sua indole polemica e della quasi nulla propensione a scendere a compromessi con gli altri membri.

\section{Il raduno}
\subsection*{21/07/2019}



Il Club di AJG decise di radunarsi a Firenze: non era abituale per il club rinuirsi, a dire il vero non era mai successo prima e non sarebbe mai dovuto succedere.

Quel maledetto giorno cambiò tutto.

Con il sorgere del Sole Firenze iniziò a brillare sotto gli occhi dei primi arrivati, tra cui Federica e Rora, un'altra ragazza del gruppo. 

In quelle ore Sirio era in volo dalla Spagna, Mancini arrivava in macchina da Milano, dove viveva, mentre Nicola e Salvo si trovavano già in città da qualche giorno per altri motivi; Sarcina, ovviamente, dormiva.

Il raduno era fissato alle 15 per favorire l'arrivo e la sistemazione di tutti i partecipanti; alle ore 12 circa Gianmarco era alle porte di Firenze, quando decise di andare a vedere il Franchi. 

"Rega io sto qua, vado a vede sto stadio, qualcuno viene o me mandate da solo?" era il messaggio con cui avvisò il gruppo delle sue intenzioni, e i pochi del gruppo che erano già riusciti a sistemarsi andarono a cercare un parcheggio per aiutarlo.

Questo scatenò la reazione di Gianluha che, con fare altezzoso, attaccò Mancini: "Senti brutto scemo avanzo di niente è inutile che provi a smentire quello che dico sta città non ha parcheggi faresti meglio a parcheggiare a Bibbiena pivello". Insomma, Sarcina si era svegliato.

Quello che il fiorentino non poteva aspettarsi era ciò che però sarebbe successo di lì a poco: nessuno era riuscito a trovare un buco libero, finché il Romano non informò il gruppo, con un messaggio, sulla sua posizione. 

Essendo i partecipanti tutti a pochi minuti a piedi, gli amici di Mancini si diressero verso il punto indicato da Mancini: chi da Ovest come Sirio, chi dalla parte orientale come Federica e Rora; Nicola e Salvo erano gli unici a mancare.

Ad un certo punto qualcosa ruppe il silenzio: era la voce di Federica da un centinaio di metri di distanza: "Oddio godo, hai trovato un parcheggio vicino a casa di Sarcina!"

Tuttavia qualcosa non quadrava: Rora sembrava turbata da qualcosa, Sirio lo colse e non si avvicinò. 

Successe tutto in pochi istanti: Sarcina girò l'angolo convinto di poter sfottere Mancini per la storia del parcheggio - d'altro canto non sapeva se quella posizione fosse una richiesta disperata di aiuto o un modo di condividere la gioia di aver effettivamente trovato un insperato posto - ma si trovò davanti una scena del tutto diversa.

Mentre Federica abbracciava Gianmarco appena sceso dall'auto la gelosia di Sarcina prese del tutto il sopravvento: Rora scappò in preda al panico ed è a Sirio, che non si mosse per cercare di capire cosa stesse succedendo, che dobbiamo la testimonianza di ciò che accadde.

Gianluha iniziò ad inveire con prepotenza verso Mancini che, inconsapevole, non si tirò certo indietro; la discussione degenerò presto e si venne alle mani: Federica tentò di scappare, ma Sarcina la placcò e dopo averla sfigurata con un coltellino per funghi la gettò sprezzante a terra.

Fu a quel punto che Mancini, inorridito dalla fine della giovane, sconcertato dalla follia negli occhi di Sarcina, terrorizzato dal destino che avrebbe atteso la città, strappò di mano la lama al vecchio amico e si tolse la vita prima che potesse farlo lui, in un estremo sacrificio; una flebile luce dorata si levò dal suo corpo inerme, sollevandosi per metri e metri e finendo per aleggiare su Firenze.

Qualche metro più lontano, Sirio osservava la scena devastato.

Una lacrima gli rigava il viso coperto dal cappuccio, quando uno squillo ruppe il silenzio.

Era un messaggio di Sarcina: "Se ti trovo fai la fine del Romano".

Sirio si allontanò rapidamente dal luogo, poi con tutta la forza che gli restava gettò a terra il telefono, mandando in mille pezzi ciò che restava del vecchio Club di AJG.

\section{Fuga da Firenze}
\subsection*{22-26/07/2019}

Sirio osservò per svariati giorni il via vai fuori dalla "Sarcina's House", così era chiamato dal gruppo l'hotel presso il quale lavorava e viveva, e decise quindi di seguire a distanza un paio di persone per capire cosa ne restasse di Mancini. Da un momento all'altro, però, lo strano bagliore sparì e da li a pochi giorni comparsero strani fenomeni, alcuni guidatori dallo sguardo assorto si erano riversati sulle strade vicino allo stadio per cercare parcheggio. La sua smania di osservare fu totalmente interrotta quando si accorse che in quella città era tutto diverso, un cadavere in strada non allertava nessuno, non volava più una parola, non si udiva alcun rumore se non il via vai delle automobili.

\subsection*{27/07/2019}
Sirio capì che non poteva trattarsi di una strana coincidenze e decise di scappare. Per fuggire aspettò un giorno tipicamente estivo in modo da poter mascherare lo sguardo con degli occhiali da sole. Salì sul primo Taxi e con voce volutamente acuta disse di voler raggiungere l'aeroporto più vicino. Il guidatore non aveva gli occhi assorti, ma non spiccicicava comunque una parola, il tassametro segnava più di 1900 euro, l'interno puzzava di putrido: fu un viaggio surreale. 
Nel tragitto notò alcuni dettagli che lo lasciarono basito, tra cui un cadavere di un marocchino riverso sulle strisce pedonali. Sirio ebbe gli occhi lucidi pensando che sarebbe potuto essere tranquillamente Isaia, un membro del club. Gli si spezzò il cuore riconoscendo Paolo, un altro ragazzo del gruppo, morto in un vicolo. Pensò ai suoi ultimi isanti probabilmente poco felici vista la composizione fecale sul muro dietro di lui. Scrisse quindi un biglietto anonimo: "@unNuovoSir", lo ripose nella tasca del sedile dietro al guidatore, poco dopo il taxi si fermò, era giunto in aeroporto. Sirio prese il primo volo per l'Oriente. Lo sguardo assente dei guidatori divenne presto iconico in tutta la città.


I mesi a seguire proseguirono in un lento declino, AJG fu distrutto e Sarcina si rivelò al mondo per ciò che veramente era: un pazzo ossessionato. In pochi anni riuscì a far diventare la sua casa un piccolo Hotel, denominato "Sarcina's House" nel quale invitava persone di rilievo che poco a poco ne uscivano diverse [Help non so come dirlo]. 
