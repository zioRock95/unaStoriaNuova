\chapter{La rovina del Mondo}
\section{Il Club di AJG}
	Le prime fonti sulla presenza di Sarcina nel mondo risalgono a questo periodo, ci giungono direttamente da Gianmarco Mancini I detto Il romano. 
Contemporanemente al periodo fiorente dello “zitto coglione” l’umanità stava inconsapevolmente dando spazio alla nascita del millennio più oscuro che si sia mai potuto documentare. Secondo fonti vicine al Romano si afferma che agli inizi dell’anno 2019 Gianluca Sarcina Il Fiorentino fu preso in giro per la totale assenza di parcheggi vicino la sua sede lavorativa a causa di una partita di calcio allo stadio Iffranchi: così veniva chiamato dal fiorentino. 

Ai tempi Sarcina era una persona simpatica a capo di un gruppo virtuale di 1000 povere anime chiamato il Club di AJG, tra cui Mancini e Sirio, un ragazzo senza nome reale, ne volto. 
Ogni tanto capitava di assistere a scherzi e momenti incresciosi nei quali Gianluha veniva accostato ad un fungo a causa del suo buffo taglio di capelli. 
Sarcina era un amministratore degno del nome, permissivo, scherzoso e ben voluto anche a causa del suo umorismo a 360 gradi. 
Anche io non mi accorsi di nulla ai tempi, ma c’erano tutti i presupposti per il peggio già allora.

E' in questo periodo che si hanno le ultime certezze sulla vita degli umani prima del grande buio. Il club di AJG rappresentava l'ultima alternativa sociale al degrado della musica Trap che ormai imperversava a ritmi serrati.
La proposta della notte, una rubrica serale dal dubbio gusto, era spesso bistrattata ed è in questi momenti che si palesò tutta la gelosia di Sarcina nei confronti di Federica, una ragazza simpatica e fidanzata che non si era mai fatta intimorire dalla molestia dei vari membri del club.
Il gruppo riusciva a non morire mai anche grazie al continuo alternarsi di temi, ironici e non, tra cui vi era anche la gelosia di Sarcina nei confronti di Mancini che riusciva a far ridere, tra gli altri, anche Federica. Molte discussioni erano animate tra Gianmarco e Salvo: un ragazzo accettato da tutti, ma compreso e non voluto a causa della sua indole polemica e della totale assenza di capacità di scendere a compromessi con gli altri membri.

\section{Il raduno}
\subsection*{21/07/2019}



Il Club di AJG decise di radunarsi a Firenze, non era abituale per il club rinuirsi, a dire il vero non era mai successo prima e non sarebbe mai dovuto succedere: quel maledetto giorno cambiò tutto.

Con il sorgere del Sole Firenze iniziò a brillare sotto gli occhi dei primi arrivati, tra cui Federica e Rora, un'altra ragazza del gruppo. In quelle ore Sirio era in volo dalla Spagna, Mancini in macchina con Nicola, Salvo si trovava già in città da qualche giorno per motivi suoi.
Sarcina dormiva, come suo solito, e l'orario del raduno era fissato alle 15.00 per favorire l'arrivo e la sistemazione di tutti i partecipanti.
Circa alle ore 12.00 Gianmarco era alle porte di Firenze, quando decide di andare a vedere lo stadio della città, ancora denominato "Il franchi". Mentre si dirigeva in quella zona avvisò il gruppo del suo imminente arrivo e i pochi del gruppo che avevano già provveduto a sistemarsi andarono a cercare un parcheggio per aiutarlo. In quel momento arrivò tuonante il messaggio di Gianluha che con fare altezzoso, ravvisò Mancini della sua ossessione nel voler sfatare ogni cosa che diceva della città, un modo come un altro per dire che si era svegliato.

Successe tutto in pochi attimi: nessuno riusciva a trovare un buco libero per posteggiare finché Il Romano fece sapere, con un messaggio di posizione, dove si trovava. Eravamo a pochi minuti a piedi, Sirio si dirigeva al punto indicato da Ovest, Rora e Federica da Est, Salvo non si presentò. Si udì la voce di Federica da un centinaio di metri di distanza: "Oddio godo, hai trovato un parcheggio vicino a casa di Sarcina!", nonostante la felicità del momento Rora sembrava turbata da qualcosa, questo dettaglio giunse all'attenzione di Sirio che non si avvicinò. 

Quel momento di gioia fu totalmente spezzato via dal terrore quando Sarcina girò l'angolo convinto di poter sfottere Mancini per la storia del parcheggio, d'altro canto nessuno sapeva se quella posizione fosse una richiesta disperata di aiuto o un modo per condividere la gioia nell'aver trovato un maledetto e osannato posto. Rombò nell'aria un urlo di disperazione non appena Federica abbracciò Gianmarco appena sceso dall'auto, la gelosia giocò un brutto scherzo a Sarcina, non si sa cosa successe di preciso, Rora scappò in preda al panico e Sirio non si mosse per cercare di capire cosa stesse succedendo, è grazie alla sua testimonianza che oggi si può dire per certo che Gianluha inveì con prepotenza contro Mancini che, inconsapevole, non si tirò certo indietro. La discussione degenerò presto e si venne alle mani: Federica tentò di scappare, ma Sarcina la placcò: dopo averla sfigurata con un coltellino per funghi la gettò sprezzante a terra, e Mancini, inorridito dalla fine della giovane, sconcertato dalla follia negli occhi di Sarcina, terrorizzato dal destino che avrebbe atteso la città, strappò di mano la lama al vecchio amico e si tolse la vita prima che potesse farlo lui, in un estremo sacrificio.

Una flebile luce dorata si levò dal suo corpo inerme, sollevandosi per metri e metri e finendo per aleggiare su Firenze.
Qualche metro più lontano, Sirio osservava la scena devastato.
Una lacrima gli rigava il viso coperto dal cappuccio, quando uno squillo ruppe il silenzio.
Era un messaggio di Sarcina: "Se ti trovo fai la fine del Romano".
Sirio chiamò un taxi per lasciare la città, poi con tutta la forza che gli restava gettò a terra il telefono, mandando in mille pezzi ciò che restava del vecchio Club di AJG.

\section{Fuga da Firenze}
\subsection*{22-26/07/2019}

Sirio osservò per svariati giorni il via vai fuori dalla Sarcina's House e decise quindi di seguire a distanza un paio di persone per capire cosa ne restava di Mancini. Da un momento all'altro, però, lo strano bagliore sparì e da li a pochi giorni comparsero strani fenomeni, alcuni guidatori dallo sguardo assorto si erano riversati sulle strade vicino allo stadio per cercare parcheggio. La sua smania di osservare fu totalmente interrotta quando si accorse che in quella città era tutto diverso, un cadavere in strada non allertava nessuno, non volava più una parola, non si udiva alcun rumore se non il via vai delle automobili.

\subsection*{27/07/2019}
Sirio capì che non poteva trattarsi di una strana coincidenze e decise di scappare. Per fuggire aspettò un giorno tipicamente estivo in modo da poter mascherare lo sguardo con degli occhiali da sole. Salì sul primo Taxi e con voce volutamente acuta disse di voler raggiungere l'aeroporto più vicino. Il guidatore non aveva gli occhi assorti, ma non spiccicicava comunque una parola, il tassametro segnava più di 1900 euro, l'interno puzzava di putrido: fu un viaggio surreale. 
Nel tragitto notò alcuni dettagli che lo lasciarono basito, tra cui un cadavere di un marocchino riverso sulle strisce pedonali. Sirio ebbe gli occhi lucidi pensando che sarebbe potuto essere tranquillamente Isaia, un membro del club. Gli si spezzò il cuore riconoscendo Paolo, un altro ragazzo del gruppo, morto in un vicolo. Pensò ai suoi ultimi isanti probabilmente poco felici vista la composizione fecale sul muro dietro di lui.
Sirio giunse in aeroporto e prese il primo volo per l'Oriente. Lo sguardo assente dei guidatori divenne presto iconico in tutta la città.


I mesi a seguire proseguirono in un lento declino, AJG fu distrutto e Sarcina si rivelò al mondo per ciò che veramente era: un pazzo ossessionato. In pochi anni riuscì a far diventare la sua casa un piccolo Hotel, denominato "Sarcina's House" nel quale invitava persone di rilievo che poco a poco ne uscivano diverse [Help non so come dirlo]. 
